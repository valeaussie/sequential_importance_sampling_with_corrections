%% Do not forget, your abstract may not be longer than 500 words.

\begin{abstract}

In this thesis we consider evolving systems for which a sequence of observations is being made, with each observation revealing additional information about current and past states of the system. We suppose each observation is made without error, but does not fully determine the state of the system at the time it is made. In particular, we look at the geographic spread of a biological invasion, where it is common to know the precise location of invasive organisms that have been detected by a surveillance program, but at any time during the program there are invaders that have not been detected.
Rebuilding the history of such partially observed invasion can help determining the real effects of past eradication efforts and help planning more effective future eradication programs.

We propose a sequential importance sampling strategy to infer the state of the invasion under a Bayesian model of such a system. The strategy involves simulating multiple alternative states consistent with current knowledge of the system, as revealed by the observations. However, a difficult problem that arises is that observations made at a later time are invariably incompatible with previously simulated states. To solve this problem, we propose a two-step iterative process in which states of the system are alternately simulated in accordance with past observations, then corrected in light of new observations. We identify criteria under which such corrections can be made while maintaining appropriate importance weights.

We also introduce a novel model for the Red Imported Fire Ants invasion in Brisbane Australia based on a self-exciting spatial temporal process and then apply our Sequential Importance Sampling methodology to infer the locations of unobserved nests each time we receive a new set of observations.
\end{abstract}