%% Do not forget, your abstract may not be longer than 500 words.

\begin{abstract}
\textcolor{blue}{(cannot exceed 500 words.)}


In this thesis We consider evolving systems for which a sequence of observations is being made, with each observation revealing additional information about current and past states of the system. We suppose each observation is made without error, but does not fully determine the state of the system at the time it is made.

For example we can consider the geographic spread of a biological invasive species, where it is common to know the precise location of invasive organisms that have been detected by a surveillance program, but at any time during the program there are invaders that have not been detected.

Rebuilding the history of a partially observed system is useful if we want to 


The spread of invasive species to new areas disrupt the equilibrium of the native ecosystem and can have adverse effects on the economy and the environment of the affected region. The real spread of the invasion is often unknown as not all individuals are detected, and this makes planning effective eradication programs difficult. The history of the invasion is usually also only partially known, so determining the real effects of past eradication efforts can be hard. 



We propose a sequential importance sampling strategy to infer the state of the invasion under a Bayesian model of such a system. The strategy involves simulating multiple alternative states consistent with current knowledge of the system, as revealed by the observations. However, a difficult problem that arises is that observations made at a later time are invariably incompatible with previously simulated states. To solve this problem, we propose a two-step iterative process in which states of the system are alternately simulated in accordance with past observations, then corrected in light of new observations. We identify criteria under which such corrections can be made while maintaining appropriate importance weights.




In this thesis we first introduce a new Sequential Importance Sampling methodology to infer the locations of unobserved nests each time we receive a new set of observations. model for the RIFA invasion based on a self-exciting spatial temporal process and then apply a novel Sequential Importance Sampling methodology to infer the locations of unobserved nests each time we receive a new set of observations.
\end{abstract}