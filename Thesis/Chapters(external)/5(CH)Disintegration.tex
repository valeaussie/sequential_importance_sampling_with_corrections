\documentclass[11pt,a4paper]{article}

\usepackage{amsmath}
\usepackage{amsthm}
\usepackage{amsfonts} 
\usepackage{graphicx} 
\usepackage[usenames]{color}
\usepackage{mathtools}
\usepackage{algorithm}
\usepackage[noend]{algpseudocode}
\usepackage{float}
\usepackage{xcolor}
\usepackage{enumitem}


\DeclarePairedDelimiter{\abs}{\lvert}{\rvert}
\DeclareMathOperator{\esssupp}{ess\,supp}


 \textwidth=16cm \hoffset = -1.9cm
 \lineskip=1.5\lineskip


% MATH -----------------------------------------------------------
\newcommand{\Real}{\mathbb R}
\newcommand{\eps}{\varepsilon}
\newcommand{\diag}{\mathrm{diag}}
\newcommand{\nbr}{\mathrm{nbr}}
\newcommand{\F}{\mathcal{F}}
\newcommand{\Hil}{\mathscr{H}}
\newcommand{\LL}{\mathcal{L}}
\newcommand{\G}{\mathscr{G}}
\newcommand{\s}{\mathbb{S}}
\newcommand{\p}{\mathscr{P}}
\newcommand{\C}{\mathscr{C}}
\newcommand{\one}[1]{\mathbf{1}_{\{#1\}}}
\newcommand{\oneset}[1]{\mathbf{1}_{#1}}
\renewcommand{\P}{\mathbb{P}}
\newcommand{\Q}{\mathsf{Q}}
\newcommand{\E}{\mathbb{E}}
\newcommand{\osimplex}{\mathcal{S}^{d-1}}
\newcommand{\csimplex}{\bar{\mathcal{S}}^{d-1}}
\newcommand{\argmin}{\mathrm{argmin}}
\newcommand{\argmax}{\mathrm{argmax}}
\newcommand{\var}{\mathrm{Var}}
\newcommand{\cov}{\mathrm{Cov}}
\newcommand{\ind}{\mathrm{I}}
\newcommand{\D}{\mathscr{D}}
\newcommand{\Borel}{\mathscr{B}}
\newcommand{\M}{\mathcal{M}}
\newcommand{\Z}{\mathcal{Z}}
\renewcommand{\d}[1]{\ensuremath{\operatorname{d}\!{#1}}}

\newcommand{\ben}{\begin{enumerate}}
\newcommand{\een}{\end{enumerate}}
\newcommand{\ds}{\displaystyle}

\DeclareMathOperator{\trace}{tr}
\DeclareMathOperator*{\esssup}{ess~sup}
\DeclareMathOperator*{\essinf}{ess~inf}
\DeclareMathOperator*{\diam}{diam}
\DeclareMathOperator*{\ROC}{ROC}
\DeclareMathOperator*{\sinc}{sinc}
\DeclareMathOperator*{\sign}{sign}
\newcommand{\voila}{\hfill $\blacksquare$}
\newcommand{\Id}{\mathrm{Id}}
\newcommand{\K}{\mathbb{K}}
\renewcommand{\Re}{\mathrm{Re}}
\renewcommand{\vec}[1]{\mathbf{#1}}
\newcommand*\diff{\mathop{}\!\mathrm{d}}

\newtheorem{theorem}{Theorem}[section]
\newtheorem{proposition}[theorem]{Proposition}
\newtheorem{corollary}{Corollary}[theorem]
\newtheorem{lemma}[theorem]{Lemma}
\newtheorem{definition}{Definition}[section]


\title{Hawkes Processes}

\begin{document}

\section{The Disintegration theorem}



\begin{thebibliography}{99}

\bibitem{Daley} Daley D and Vere-Jones D (2003) An Introduction to the Theory of Point Processes: Elementary Theory and Methods. \textit{Springer-Verlag New York}

\bibitem{Hawkes71} Hawkes A G (1971) Spectra of Some Self-Exciting and Mutually Exciting Point Processes. \textit{Biometrika} 58(1): 83-90

\bibitem{Hawkes74} Hawkes A G, Oakes D (1974) A Cluster Process Representation of a Self-Exciting Process. \textit{Journal of Applied Probability} 11(3): 493-503

\bibitem{Jewell} Jewell C P, Kypraios T, Neal P, Roberts G O (2009) Bayesian Analysis for Emerging Infectious Diseases. \textit{Bayesian Anal.} 4(3): 465-496

\bibitem{Jacobsen} Jacobsen M (2006) Point Process Theory and Applications. Marked Point and Piecewise Deterministic Processes. \textit{Birkh\"{a}user}

\bibitem{Laub} Laub P (2014) Hawkes Processes: Simulation, Estimation, and Validation. \textit{The University Of Queensland. Australia}

\bibitem{Lewis} Lewis P A W, Shedler G S (1979) Simulation of Non Homogeneous Poisson Process by Thinning. \textit{Naval Res. Logistics Quart.} 26(3): 403–413

\bibitem{Mohler} Mohler G O (2011) Self-Exciting Point Process Modeling of Crime. \textit{Journal of the American Statistical Association} 106(493): 100–108

\bibitem{Ogata81} Ogata Y (1981) Statistical On Lewis' Simulation Method for Point Processes. \textit{IEEE Transactions On Information Theory} 27(1): 23–31

\bibitem{Ogata88} Ogata Y (1988) Statistical Models for Earthquake Occurrences and Residual Analysis for Point Processes. \textit{Journal of the American Statistical Association} 83(401): 9–27

\bibitem{Ozaki} Ozaki T (1979) Maximum Likelihood Estimation of Hawkes' Self Exciting Point Processes. \textit{Annals of the Institute of Statistical Mathematics} 31 Part B: 145–155

\bibitem{Rasmussen} Rasmussen J G (2009) Temporal Point Processes: the Conditional Intensity Function. \textit{Course notes for ‘rumlige punktprocesser’ (spatial point processes)}.

\bibitem{Reinhart} Reinhart A (2017) A Review of Self-Exciting Spatio-Temporal Point Processes and their Applications. \textit{Statist. Sci.} 33(3): 299-318

\bibitem{Shoenberg} Shoenberg F P, Brillinger R, Guttorp P (2014). Point Processes, Spatial Temporal. \textit{Encyclopedia of Environmetrics}

\bibitem{Snyder} Snyder D L, Miller I (1991). Random Point Processes in Time and Space. \textit{Springer-Verlag New York}

\bibitem{Zhuang} Zhuang J, Ogata Y, Vere-Jones D (2004). Analyzing Earthquake Clustering Features by Using Stochastic Reconstruction. \textit{Journal of Geophysical Research: Solid Earth} 109(B05301)

\end{thebibliography}


\end{document}
