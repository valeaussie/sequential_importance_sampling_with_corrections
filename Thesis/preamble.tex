

%\usepackage[a4paper,inner=2.7cm,top=2.3cm,outer=2cm,bottom=2cm]{geometry}
\usepackage[a4paper,left=2.35cm,top=2.3cm,right=2.35cm,bottom=2cm]{geometry}
\usepackage{fancyhdr}
\pagestyle{fancy}

\fancyhead[LO]{}
\fancyhead[RE]{}
\fancyhead[RO]{\slshape \rightmark}
\fancyhead[LE]{\slshape \leftmark}
\fancyfoot[C]{\thepage}
\setlength{\headheight}{15pt}

\makeatletter
\newcommand\frontmatter{%
  \cleardoublepage
  \pagenumbering{roman}
}

\newcommand\mainmatter{%
  \cleardoublepage
  \pagenumbering{arabic}
}

\newcommand\backmatter{%
  \if@openright
    \cleardoublepage
  \else
    \clearpage
  \fi
  \pagenumbering{alpha}
}
\makeatother

%%%%%%%%%%%%%%%%%%%%%%%%%%%%%%%%%%%%%%%%%%%%%%%%%%%%%%%%
% Packages (alphabetical order)
%%%%%%%%%%%%%%%%%%%%%%%%%%%%%%%%%%%%%%%%%%%%%%%%%%%%%%%%

\usepackage{algorithm}
\usepackage[noend]{algpseudocode}
\usepackage{amsfonts}
\usepackage{amsmath}
\usepackage{amssymb}
\usepackage{amstext}
\usepackage{amsthm}
\usepackage{array}
\usepackage{authblk}
\usepackage{booktabs}
\usepackage[colorlinks=true, linkcolor = blue, citecolor = dgreen]{hyperref} % Needs to be loaded before cleveref. If you don't like hyperlinks comment this out!
\usepackage{cleveref}
%[noabbrev,capitalise]
\usepackage[usenames,dvipsnames]{xcolor}
\usepackage{colortbl}
\usepackage{emptypage}
\usepackage[shortlabels]{enumitem}
\usepackage[english]{babel}
\usepackage{enumitem}
\usepackage{eqparbox}
\usepackage{extarrows}
\usepackage{eucal}
\usepackage{float}
%\usepackage{graphicx} \graphicspath{ {img/} }
\usepackage[demo]{graphicx}
\usepackage{latexsym}
%-----------------------------------------------------------------------I dunno what these two packages do but they wouldn't run on my PC - Kaustav
%\usepackage{listings}
%\usepackage{lstlinebgrd}
\usepackage{marvosym}
\usepackage{mathrsfs}
\usepackage{mathtools}
\usepackage{multirow}
\usepackage{natbib}
\usepackage{setspace}
\usepackage{soul}
\usepackage{subfigure}
\usepackage{todonotes}
%\usepackage{tablefootnote}
\usepackage{tikz}
  \usetikzlibrary{matrix,calc}
  \usetikzlibrary{decorations.markings,decorations.pathreplacing}
  \usetikzlibrary{patterns}
\usepackage[usenames]{color}
\usepackage{url}
\usepackage{xcolor}
\usepackage{xspace}


%%%%%%%%%%%%%%%%%%%%%%%%%%%%%%%%%%%%%%%%%%%%%%%%%%%%%%%%
% If you prefer the slanted inequalities (Kaustav does not)
%%%%%%%%%%%%%%%%%%%%%%%%%%%%%%%%%%%%%%%%%%%%%%%%%%%%%%%%
%\renewcommand{\geq}{\geqslant}
%\renewcommand{\leq}{\leqslant}
%\renewcommand{\ge}{\geqslant}
%\renewcommand{\le}{\leqslant}

%%%%%%%%%%%%%%%%%%%%%%%%%%%%%%%%%%%%%%%%%%%%%%%%%%%%%%%%
% I prefer ``(1)'' to ``Equation (1)'' from cleveref. Update from Kaustav, 'I don't'.
%%%%%%%%%%%%%%%%%%%%%%%%%%%%%%%%%%%%%%%%%%%%%%%%%%%%%%%%
%\crefname{equation}{}{}
%\Crefname{equation}{}{}

%%%%%%%%%%%%%%%%%%%%%%%%%%%%%%%%%%%%%%%%%%%%%%%%%%%%%%%%
% Useful commands
%%%%%%%%%%%%%%%%%%%%%%%%%%%%%%%%%%%%%%%%%%%%%%%%%%%%%%%%

\newcommand{\etal}{\emph{et al.}\xspace}
\newcommand{\CITE}{{\bf {\color{red} [CITE]}}\xspace}
\newcommand{\REF}{{\bf {\color{red} [REF]}}\xspace}
\newcommand{\FIGURE}{{\bf {\color{red} [FIG]}}}
\newcommand{\NOTE}[1]{\renewcommand{\fboxsep}{1mm}\framebox{\textsf{\em #1}}\marginpar{\bf$\leftarrow$NOTE}}
\newcommand{\CONTENT}{{\medskip \center{\Huge \ldots} \marginpar{\bf$\leftarrow$MORE} \medskip}}
\newcommand{\bigO}[1]{\mathcal{O}(#1)\xspace}
\newcommand{\littleo}[1]{o(#1)\xspace}
\newcommand{\littleoo}[1]{o_0(#1)\xspace}
\newcommand{\Gram}{\text{Gram}}
\newcommand{\tr}{\text{Tr}}
\newcommand{\aip}[2]{| \langle #1 , #2 \rangle |}
\newcommand{\BR}{\{e^{\frac{2\pi i}{3}},e^{-\frac{2\pi i}{3}}\}}
\renewcommand{\bar}[1]{\overline{#1}}    % Overline looks nicer for conjugation...
\newcommand{\con}{\rightarrow\leftarrow} % Contradiction

\newcommand{\Real}{\mathbb R}
\newcommand{\eps}{\varepsilon}
\newcommand{\diag}{\mathrm{diag}}
\newcommand{\nbr}{\mathrm{nbr}}
\newcommand{\F}{\mathcal{F}}
\newcommand{\Hil}{\mathscr{H}}
\newcommand{\LL}{\mathcal{L}}
\newcommand{\G}{\mathscr{G}}
\newcommand{\s}{\mathbb{S}}
\newcommand{\p}{\mathscr{P}}
\newcommand{\C}{\mathscr{C}}
\newcommand{\one}[1]{\mathbf{1}_{\{#1\}}}
\newcommand{\oneset}[1]{\mathbf{1}_{#1}}
\renewcommand{\P}{\mathbb{P}}
\newcommand{\Q}{\mathsf{Q}}
\newcommand{\E}{\mathbb{E}}
\newcommand{\osimplex}{\mathcal{S}^{d-1}}
\newcommand{\csimplex}{\bar{\mathcal{S}}^{d-1}}
\newcommand{\argmin}{\mathrm{argmin}}
\newcommand{\argmax}{\mathrm{argmax}}
\newcommand{\var}{\mathrm{Var}}
\newcommand{\cov}{\mathrm{Cov}}
\newcommand{\ind}{\mathrm{I}}
\newcommand{\D}{\mathscr{D}}
\newcommand{\Borel}{\mathscr{B}}
\newcommand{\M}{\mathcal{M}}
\newcommand{\Z}{\mathcal{Z}}
\renewcommand{\d}[1]{\ensuremath{\operatorname{d}\!{#1}}}
%\renewcommand{\d}[1]{\ensuremath{\operatorname{d}\!{#1}}}
\newcommand{\ben}{\begin{enumerate}}
\newcommand{\een}{\end{enumerate}}
\newcommand{\ds}{\displaystyle}
\newcommand{\voila}{\hfill $\blacksquare$}
\newcommand{\Id}{\mathrm{Id}}
\newcommand{\K}{\mathbb{K}}
\renewcommand{\Re}{\mathrm{Re}}
\renewcommand{\vec}[1]{\mathbf{#1}}
\newcommand*\diff{\mathop{}\!\mathrm{d}}

% Nicer looking gray
\definecolor{Gray}{gray}{0.9}
%---- A better green - Kaustav
\definecolor{dgreen}{rgb}{0,.8,0}

%------------------------------------------------------------------------Darcy's preferred theorem etc environment----------------------------------------------------------------
%-----------------------------------------------------------------------------------------------------------------------------------------------------------------------------------------
%-----------------------------------------------------------------------------------------------------------------------------------------------------------------------------------------
%%%%%%%%%%%%%%%%%%%%%%%%%%%%%%%%%%%%%%%%%%%%%%%%%%%%%%%%
%\theoremstyle{definition}
%\newtheorem{theorem}{Theorem}[chapter]
%\newtheorem{lemma}[theorem]{Lemma}
%\newtheorem{corollary}[theorem]{Corollary}
%\newtheorem{proposition}[theorem]{Proposition}
%
%\theoremstyle{definition}
%\newtheorem{definition}[theorem]{Definition}
%\newtheorem{conjecture}[theorem]{Conjecture}
%\newtheorem{example}[theorem]{Example}
%\newtheorem{assumption}{Assumption}[section]
%
%\Crefname{conjecture}{Conjecture}{Conjectures} %% It doesn't know how to pluralize Conjecture
%
%\theoremstyle{remark}
%\newtheorem{remark}[theorem]{Remark}
%\newtheorem{notation}[theorem]{Notation}
%\newtheorem{note}[theorem]{Note}
%\newtheorem{summary}[theorem]{Summary}
%\newtheorem{problem}[theorem]{Problem}
%\newtheorem{construction}[theorem]{Construction}

%---------------------------------------------------------------Kaustav's preferred theorem etc environment----------------------------------------------------------------------
%-----------------------------------------------------------------------------------------------------------------------------------------------------------------------------------------
%-----------------------------------------------------------------------------------------------------------------------------------------------------------------------------------------
\theoremstyle{definition}
\newtheorem{theorem}{Theorem}[section]
\newtheorem{question}{Question}
\newtheorem{definition}{Definition}[section]
\newtheorem{proposition}{Proposition}[section]
\newtheorem{remark}{Remark}[section]
\newtheorem{assumption}{Assumption}[section]
\newtheorem{lemma}{Lemma}[section]
\newtheorem{corollary}{Corollary}[section]
\newtheorem{exercise}{Exercise}[section]



% I like using arrays to hold equations (lines up the '=' nicely). Update from Kaustav, `I don't'.
% They need to be vertically spaced different than a normal matrix, though
% When you make an array for equation, do something like this:
%   \equationarray
%   \begin{array}{lcr}
%     f(x) & = & (x-2)^2 \\
%          & = & x^2 - 4x + 4
%   \end{array}
%   \normalarray
\newcommand{\equationarray}{\renewcommand{\arraystretch}{1.55}}
\newcommand{\normalarray}{\renewcommand{\arraystretch}{1.1}}
\newcommand{\appendixarray}{\renewcommand{\arraystretch}{1}}

\normalarray

%%%%%%%%%%%%%%%%%%%%%%%%%%%%%%%%%%%%%%%%%%%%%%%%%%%%%%%%
%  Common mathematical sets.
%%%%%%%%%%%%%%%%%%%%%%%%%%%%%%%%%%%%%%%%%%%%%%%%%%%%%%%%

\newcommand{\A}{\mathbb{A}} % Algebraic Numbers
\newcommand{\C}{\mathbb{C}} % Complex
\newcommand{\F}{\mathbb{F}} % Field
%\newcommand{\N}{\mathbb{N}} % Naturals
\newcommand{\naturals}{\mathbb{N}} % Naturals (Kaustav's preferred syntax)
\newcommand{\Q}{\mathbb{Q}} % Rationals
%\newcommand{\R}{\mathbb{R}} % Reals
\newcommand{\reals}{\mathbb{R}} % Reals (Kaustav's preferred syntax)
\newcommand{\exreals}{\mathbb{ \overline R}} % Extended reals
\newcommand{\T}{\mathbb{T}} % Unimodular
%\newcommand{\Z}{\mathbb{Z}} % Integers
\newcommand{\integers}{\mathbb{Z}} % Integers (Kaustav's preferred syntax)


%%%%%%%%%%%%%%%%%%%%%%%%%%%%%%%%%%%%%%%%%%%%%%%%%%%%%%%%
%  Fancy letters
%     (Note that A and S are missing --
%            \AA and \SS are something else)
%%%%%%%%%%%%%%%%%%%%%%%%%%%%%%%%%%%%%%%%%%%%%%%%%%%%%%%%
\newcommand{\alt}{\mathcal{A}} % Alternating group
\newcommand{\BB}{\mathcal{B}}
\newcommand{\CE}{\mathcal{C}}
\newcommand{\DD}{\mathcal{D}}
\newcommand{\EE}{\mathcal{E}}
\newcommand{\FF}{\mathcal{F}}
\newcommand{\GG}{\mathcal{G}}
\newcommand{\HH}{\mathcal{H}}
\newcommand{\II}{\mathcal{I}}
\newcommand{\JJ}{\mathcal{J}}
\newcommand{\KK}{\mathcal{K}}
\newcommand{\LL}{\mathcal{L}}
\newcommand{\MM}{\mathcal{M}}
\newcommand{\NN}{\mathcal{N}}
\newcommand{\OO}{\mathcal{O}}
\newcommand{\PP}{\mathcal{P}}
\newcommand{\QQ}{\mathcal{Q}}
\newcommand{\RR}{\mathcal{R}}
\newcommand{\TT}{\mathcal{T}}
\newcommand{\UU}{\mathcal{U}}
\newcommand{\VV}{\mathcal{V}}
\newcommand{\WW}{\mathcal{W}}
\newcommand{\XX}{\mathcal{X}}
\newcommand{\YY}{\mathcal{Y}}
\newcommand{\ZZ}{\mathcal{Z}}

%%%%%%%%%%%%%%%%%%%%%%%%%%%%%%%%%%%%%%%%%%%%%%%%%%%%%%%%
%  Common math operators
%%%%%%%%%%%%%%%%%%%%%%%%%%%%%%%%%%%%%%%%%%%%%%%%%%%%%%%%
\DeclareMathOperator{\per}{per} %Update from Kaustav, `yeah super common'.
\DeclareMathOperator{\adj}{Adj}
\DeclareMathOperator{\rank}{rank}


%%%%%%%%%%%%%%%%%%%%%%%%%%%%%%%%%%%%%%%%%%%%%%%%%%%%%%%%
%  Probability/measure theory
%%%%%%%%%%%%%%%%%%%%%%%%%%%%%%%%%%%%%%%%%%%%%%%%%%%%%%%%
\newcommand{\Pro}{\mathbb{P}}
\newcommand{\Qro}{\mathbb{Q}}
\newcommand{\Ex}{\mathbb{E}}
\newcommand{\Var}{\mathrm{Var}}
\newcommand{\Cov}{\mathrm{Cov}}
\newcommand{\Corr}{\mathrm{Corr}}
\newcommand{\borel}{\mathcal{B}(\mathbb{R})}

%%%%%%%%%%%%%%%%%%%%%%%%%%%%%%%%%%%%%%%%%%%%%%%%%%%%%%%%
%  Convenience
%%%%%%%%%%%%%%%%%%%%%%%%%%%%%%%%%%%%%%%%%%%%%%%%%%%%%%%%
\newcommand{\dd}{\mathrm {d}}
\newcommand{\ep}{\epsilon}
\newcommand{\vep}{\varepsilon}

%%%%%%%%%%%%%%%%%%%%%%%%%%%%%%%%%%%%%%%%%%%%%%%%%%%%%%%%%
%% Latin terms (i.e., e.g.,
%%%%%%%%%%%%%%%%%%%%%%%%%%%%%%%%%%%%%%%%%%%%%%%%%%%%%%%%%
\newcommand{\ie}{\emph{i.e., }\xspace}
\newcommand{\eg}{\emph{e.g., }\xspace}
\newcommand{\cf}{\emph{cf. }\xspace}

%%%%%%%%%%%%%%%%%%%%%%%%%%%%%%%%%%%%%%%%%%%%%%%%%%%%%%%%
%  A different look for \begin{list}
%%%%%%%%%%%%%%%%%%%%%%%%%%%%%%%%%%%%%%%%%%%%%%%%%%%%%%%%
\newenvironment{myind}[1]%
 {\begin{list}{}%
    {\setlength{\leftmargin}{#1}}%
  \item[]%
 }
{\end{list}}

%%%%%%%%%%%%%%%%%%%%%%%%%%%%%%%%%%%%%%%%%%%%%%%%%%%%%%%%
%  Look for Latin squares
%%%%%%%%%%%%%%%%%%%%%%%%%%%%%%%%%%%%%%%%%%%%%%%%%%%%%%%%
\tikzset{square matrix/.style={
    matrix of nodes,
    column sep=-\pgflinewidth, row sep=-\pgflinewidth,
    nodes={draw,
      minimum height=15pt,
      anchor=center,
      text width=13pt,
      align=center,
      inner sep=0pt
    },
  },
  square matrix/.default=2cm
}

\colorlet{trans}{blue!50}
\colorlet{trans2}{red!20}



\newcommand{\X}{\times\xspace}
\newcommand{\x}{$\X$\xspace}
\newcommand{\BLAH}{{\color{red} BLAH}}


%%%%%%%%%%%%%%%%%%%%%%%%%%%%%%%%%%%%%%%%%%%%%%%%%%%%%%%%%
%% Algorithm stuff
%%%%%%%%%%%%%%%%%%%%%%%%%%%%%%%%%%%%%%%%%%%%%%%%%%%%%%%%%

 
\algnewcommand\algorithmicforeach{\textbf{for each}}
\algdef{S}[FOR]{ForEach}[1]{\algorithmicforeach\ #1\ \algorithmicdo}

\newcommand{\myComment}[1]{\Comment{{\color{blue} #1}}}

%%%%%%%%%%%%%%%%%%%%%%%%%%%%%%%%%%%%%%%%%%%%%%%%%%%%%%%%%
%% What should the code look like?
%%%%%%%%%%%%%%%%%%%%%%%%%%%%%%%%%%%%%%%%%%%%%%%%%%%%%%%%%
%\newcounter{lnum}
%\lstset{
%  language=C++,
%  basicstyle=\scriptsize\ttfamily\bfseries,
%  numbers=none, 
%  numberstyle=\tiny, 
%%  stepnumber=5,
%  linebackgroundsep={0.11in},
%  keywordstyle=\color{blue}, 
%  commentstyle=\color{red},
%  stringstyle=\ttfamily, % typewriter type for strings
%  %frame=bt, 
%  %rulesepcolor=\color{red!20!green!20!blue!20},
%  %rulecolor=\color{Brown},
%  escapeinside=``, 
%  %xleftmargin=2em,xrightmargin=2em, aboveskip=1em,
%  breaklines=true             % sets automatic line breaking
%}

%%%%%%%%%%%%%%%%%%%%%%%%%%%%%%%%%%%%%%%%%%%%%%%%%%%%%%%%%
%% Cover paper stuff
%%%%%%%%%%%%%%%%%%%%%%%%%%%%%%%%%%%%%%%%%%%%%%%%%%%%%%%%%

\def\Urcs{\mathcal{U}_\mathrm{RCS}}
\def\Urc{\mathcal{U}_\mathrm{RC}}
\def\Urs{\mathcal{U}_\mathrm{RS}}
\def\Ucs{\mathcal{U}_\mathrm{CS}}
\def\Ur{\mathcal{U}_\mathrm{R}}
\def\Uc{\mathcal{U}_\mathrm{C}}
\def\Us{\mathcal{U}_\mathrm{S}}
\def\Vr{\mathcal{V}_\mathrm{R}} % Used in one proof
\def\Vc{\mathcal{V}_\mathrm{C}} %
\def\Vs{\mathcal{V}_\mathrm{S}} %

\newcommand{\unkn}{\bullet}
\newcommand{\pmax}{p_{\text{\normalfont max}}}
\newcommand{\qmin}{q_{\text{\normalfont min}}}

\def\L{\mathscr{L}}
\def\G{\mathscr{G}}
\def\cov{\mathscr{C}}


\setcitestyle{authoryear,open={(},close={)}}