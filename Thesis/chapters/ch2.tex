\chapter[Hawkes Processes and Spatio-temporal self-exciting point processes]{Hawkes Processes and Spatio-temporal self-exciting point processes}
\label{ch:Hawkes}

The idea of self-exciting point processes was first introduced by Hawkes in 1971 \cite{Hawkes71} and has been since applied to a variety of fields like economic sciences, natural sciences, social sciences and more. These models are well suited to events that would cluster, like for example the aftershocks following earthquakes \cite{Ogata88}, \cite{Mohler}. We will utilise a self-exciting spatial-temporal point process to model the RIFA invasion as in general invasive species involve clustering. This model will alos allow us to disregard the phylogeny of the nests making the simulations faster than for some existing models for this invasion. The phylogeny could be reconstructed if needed.

In this chapter we aim to introduce self-exciting point process in a general manner. This is not an in depth presentation of the subject which can be found in several books like Daily and Vere-Jones ``An Introduction to the Theory of Point Processes: Elementary Theory and Methods" \cite{Daley} or Snyder``Random Point Process in Time and Space" \cite{Snyder}.

A self-exciting point process is a counting process modelling a sequence of events over time. Each of these events will ``excite" the process, meaning that the likelihood of a new event is increased for some time after the first arrival. We will first consider the one dimensional case also known as Hawkes process.

Let us first define a counting process as a stochastic process which represents the cumulative count of the number of events in the time interval $[0, t]$. More formally

\begin{definition}
    A \textbf{counting process} is a stochastic process $(N(t) : t \geq 0)$ such that:
    \begin{enumerate}[label=(\roman*)]
        \item $N(t)$ is integer valued.
        \item If $s<t$, then $N(s) \leq N(t)$.
    \end{enumerate}
\end{definition}

This can be also viewed as a simple point process characterised by the sequence of random arrivals times $\vec{T} = \{ t_1, t_2 \dots \}$ at which the counting process has jumped. More formally (definition adapted from Jacobsen \cite{Jacobsen} chapter 2.1)

\begin{definition}
    A \textbf{simple point process} is a sequence of random variables $\vec{T} = \{ t_1, t_2 \dots \}$ defined on the probability space $(\Omega, \mathcal{F}, \mathbb{P})$ and with values in $\mathbb{R}_{>0} \cup \infty$ such that:
    \begin{enumerate}[label=(\roman*)]
        \item $\mathbb{P}(0 < t_0 \leq t_1 \leq \dots) = 1$.
        \item $\mathbb{P}(t_n < t_{n+1}, t_n < \infty) = \mathbb{P}(t_n < \infty)$ for $n \geq 0$.
        \item The number of random variables in a bounded region is finite almost surely (a.s.).
    \end{enumerate}
\end{definition}

Hence a simple point process is an almost surely increasing sequence of strictly positive, possibly infinite random variables, strictly increasing as long as they are finite and such that only finitely many events can occur in any finite time interval. Notice that no points coincide as they are strictly ordered in time.
The interpretation of a finite $t_n$ is the timepoint at which the $n$th recording of an event takes place with less than $n$ events occurring all together on the time axis $\mathbb{R}_{>0} \cup \infty$ if $t_n = \infty$. By definition no event can happen at time $0$. See Daley and Vere-Jones chapter 3 for a complete explanation on how Point processes on the line can be described \cite{Daley}.

We can characterise a point process defining the cumulative distribution function of the next arrival time conditioned on the past events $\mathcal{H}_u$

\begin{equation} \label{eq:CumDistPointProc}
    F(t|\mathcal{H}_u) = \int_u^t \mathbb{P}(t_{n+1} \in [s+\d s] | \mathcal{H}_u) \d s = \int_u^t f(s | \mathcal{H}_u) \d s
\end{equation}
were $f(s | \mathcal{H}_u)$ is the probability density function.

By the chain rule the joint density for a realisation $\{ t_1, \dots, t_k\}$ will therefore be

\begin{equation}\label{eq:JointDensHawkes}
    p(t_1, \dots, t_k) = \prod_{i=1}^k f(t_i | \mathcal{H}_{t_{i-1}}).
\end{equation}
A point process is completely defined if the joint probability distributions are known for the number of events in all finite families of disjoint intervals (see Daley and Vere-Jones chapter 5 and 9 \cite{Daley}).

If a point process has a probability distribution function $f(t)$ that is independent of the history $\mathcal{H}_t$ we call this process a \textit{renewal process}. Another way to state this is that $f(t) = g(t - t_k)$ for some p.d.f. $g : \mathbb{R}^+ \rightarrow \mathbb{R}^+$. This way the interarrival times are independent and identically distributed (i.i.d) random variables.

An example of renewal process is the \textit{homogeneous Poisson Process}

\begin{definition}\label{def:HomPoi}
    Consider a counting process $(N(t):t \geq 0)$ over the half-open interval $(a_i, b_i]$ with $a_i < b_i \leq a_{i+1}$, then a \textit{stationary or homogeneous Poisson Process} on the line is completely defined by
    \begin{equation}\label{001}
        \mathbb{P}(\{N(a_i, b_i] = n, i = 1, \dots, k \}) = \prod_{i=1}^k \frac{[\lambda(b_i - a_i)]^{n_i}}{n_i!}e^{-\lambda(b_i-a_i)}.
    \end{equation}
\end{definition}
    
Therefore for an homogeneous Poisson process the number of points in each finite interval $(a_i, b_i]$ has a Poisson distribution. Also, the number of points in disjoint intervals are independent random variables and the distributions are stationary, meaning that they depend only on the lengths of the intervals $b_i - a_i$.

The likelihood of a finite Poisson Process is the probability of obtaining the given number of observations $N$ in the observation period $(0,T]$, times the joint conditional density for the locations of those observations $t_1, \dots, t_N$, given their number. From equation (\ref{001}) we see that the probability of obtaining one event in $(t_i - \Delta, t_i]$ and no points in the remaining parts of $(0, T]$ is

\begin{equation*}
    e^{-\lambda T} \prod_{i=1}^N \lambda \Delta.
\end{equation*}
Dividing by $\Delta^N$ and letting $\Delta \rightarrow 0$ we obtain the density and therefore find the likelihood

\begin{equation*}
    L_{(0, T]}(N;t_1, \dots, t_N) = \lambda^N e^{-\lambda T}.
\end{equation*}
A Poisson process with time-varying rate $\lambda(t)$ is called \textit{inhomogeneous} Poisson process and can be defined the same way as the homogeneous Poisson process in Definition \ref{def:HomPoi} with the quantities $\lambda(b_i - a_i) = \int_{a_i}^{b_i} \lambda \d x$ replaced by

\begin{equation*}
    \Lambda(a_i, b_i] = \int_{a_i}^{b_i} \lambda(x) \d x.
\end{equation*}
The likelihood of a inhomogeneous Poisson process will then be

\begin{equation} \label{eq:LikInhPoiPro}
    L_{(0, T]}(N;t_1, \dots, t_N) = e^{-\Lambda(0, T]}\prod_{i=1}^N \lambda(t_i)
\end{equation}
and since $N$ is a counting process

\begin{equation*}
    e^{-\Lambda(0, T]}\prod_{i=1}^N \lambda(t_i) = \exp \Bigg( - \int_0^T \lambda(t)\d t + \int_0^T \log \lambda(t) N (\d t) \Bigg).
\end{equation*}
Let us consider now  a time change $t \rightarrow u(t) \equiv \Lambda(0, t]$. For all $t \geq 0$ we write $N(t) = N (0, t]$ and define a new point process by $\tilde{N}(t) = N(u^{-1}(t))$. It follows that this new process has a rate quantity $\tilde{\Lambda}(0, t) = u(u^{-1}(t)) = t$ and is therefore a stationary Poisson process with $\lambda = 1$.

It can be shown that any point process that stratifies certain continuity condition can be transformed into a Poisson  process. 

We can now define the self-exciting Hawkes model. This model combines a cluster process representation and a simple conditional intensity representation which is moreover linear.

\begin{definition}\label{def:Hawkes}
    Let us consider a point process $N(t)$ such that
    \begin{enumerate}[label=(\roman*)]
        \item $\mathbb{P}(\{ N(t+h) - N(t) = 1 \} | \mathcal{H}_t) = \lambda(t)h + o(h)$
        \item $\mathbb{P}(\{ N(t+h) - N(t) > 1 \} | \mathcal{H}_t) = o(h)$
    \end{enumerate}
    If the quantity $\lambda(t)$ can be written as
    \begin{equation*}
        \lambda(t | \mathcal{H}_t) = \mu + \int_{0}^t g(t - u) \d N(u)
    \end{equation*}
    for some $\mu \in \mathbb{R^+}$ and $g : \mathbb{R^+} \rightarrow \mathbb{R^+} \cup \{0\}$ then this process is a \textbf{Hawkes process}.
\end{definition}
The quantity $\mu$ is called \textit{background intensity} and the function $g$ is called \textit{excitation function}. The quantity $\lambda(t | \mathcal{H}_t)$ representing the rate of events at time $t$ is called \textit{conditional intensity} and can be also written as

\begin{equation*}
    \lambda(t|\mathcal{H}_t) = \mu + \sum_{t_i < t} g(t - t_i)
\end{equation*}
where $\{ t_1, \dots, t_k\}$ represents the observed sequence of past events up to time $t$.

The process is called ``self-exciting" because of the dependence of the conditional intensity function on the past history $\mathcal{H}_t$ of the system allowing events to trigger new events. The triggering function $g$ can take many forms, so the process may depend only on the recent history if $g$ decays rapidly or may have longer term effects \cite{Reinhart}. However, we generally require that $\lambda(t | \mathcal{H}_t)  \geq 0$, therefore we must have $g(t) \geq 0$ for every value of $t$ and $g(t) = 0$ for $t < 0$.

The general definition for the conditional intensity $\lambda$ for any point process is

\begin{equation*}
    \lambda(t | \mathcal{H}_t) = \lim_{h \to 0} \frac{\mathbb{E}[N(t+h) - N(t) | \mathcal{H}_t]}{h}
\end{equation*}
so it represents the mean number of events in a region conditional on the past.
Let us rewrite this last equation as \cite{Rasmussen}

\begin{equation*}
    \lambda(t | \mathcal{H}_t) \d t = \mathbb{E}[N(\d t) | \mathcal{H}_t]
\end{equation*}
were $\d t$ is an infinitesimal interval around $t$.
The intensity function will therefore be:

\begin{align*}
    \mathbb{E}[N(\d t) | \mathcal{H}_t] &= \mathbb{P}(\textrm{point in} \d t | \mathcal{H}_t) \\
    & = \mathbb{P}(\textrm{point in} \d t | \textrm{point not before } t, \mathcal{H}_t) \\
    & = \frac{\mathbb{P}(\textrm{point in} \d t, \textrm{point not before } t | \mathcal{H}_t)}{\mathbb{P}(\textrm{point not before } t | \mathcal{H}_t))} \\
    & = \frac{\mathbb{P}(\textrm{point in} \d t | \mathcal{H}_t)}{\mathbb{P}(\textrm{point not before } t | \mathcal{H}_t))} \\
    & = \frac{f(t) \d t}{1 - F(t)}
\end{align*}
Where the quantities $f(t)$ and $F(t)$ where defined in equation (\ref{eq:CumDistPointProc}). So the intensity function can be written
\begin{equation}\label{eq:IntFun}
    \lambda(t) = \frac{f(t) \d t}{1 - F(t)}.
\end{equation}
For ease of notation here and from now on we are omitting the conditioning over the past events $\mathcal{H}_t$.

It has been proved by Hawkes and Oakes \cite{Hawkes74} that any stationary self-exciting point process with finite intensity function can be interpreted as a Poisson cluster process. Then events are partitioned into two disjoint processes: a background process for the cluster centers which is a Poisson process with rate $\mu$, and separate offspring process of the events triggered inside each cluster, whose intensities are determined by $g$. Each triggered event may then trigger new events. The number of offspring of each event is drawn from a Poisson distribution with mean 

\begin{equation*}
    m = \int_0^\infty g(t) \d t.
\end{equation*}

So far we have considered only the temporal form of the self-exciting point processes but we can extend this models to spatio-temporal processes. The definition will be analogous to Definition \ref{def:Hawkes} with conditional intensity

\begin{equation*}
    \lambda(s, t) = \mu(s) + \sum_{i:t_i<t} g(s - s_i, t - t_i),
\end{equation*}
where $\{ s_1, \dots, s_n \}$ are the observed locations of events and $\{t_1, \dots t_n \}$ the observed times of events. 

Similarly to the Hawkes' self-exciting temporal processes, the self-exciting spatio-temporal point processes can be treated as Poisson cluster processes with the locations of events $s \in X \subseteq \mathbb{R}^d$ and times $t \in (0, T]$. The mean number of offspring will be

\begin{equation}\label{eq:meanoffsp}
    m = \int_X \int_0^T g(s, t) \d t \d s.
\end{equation}
So far we have considered the points of a point process as indistinguishable other then by their time and location. Often, however, we need to take in consideration other quantities. For example, one may wish to analyse locations and time of observations of an invasive species plant alongside the age of the organism. Such process may be viewed as a \textit{marked} point process. A marked point process is a point process of events $\{ (s_i, t_i, k_i ) \}$ where  $s_i \in X \subseteq \mathbb{R^d}$, $t_i \in [0, T)$ and $k_i \in \mathcal{K}$ where $K$ is the \textit{marked space} (in our example the space of plants ages).

The \textit{ground process} of a marked point process is the point process of events locations and times without the marks with conditional intensity $\lambda_g(s, t)$. The conditional intensity for the marked point process can be written as

\begin{equation*}
    \lambda(s, t, k) = \lambda_g(s, t)f(k | s, t),
\end{equation*}
where $f(k | s, t)$ is the conditional density of the mark at time $t$ and location $s$ given the history of the process up to time $t$.

\section{The likelihood of a self-exciting point process}

We obtain the Likelihood for a Hawkes process from the following theorem {\color{missing a little introduction parhaps?}}

\begin{theorem}
    Let $N(\cdot)$ be a regular point process on [0, T] for some positive T, and let $\{ t_1, \dots, t_k \}$ denote the realisation of $N(\cdot)$ over [0, T]. Then, the likelihood $L$ of $N(\cdot)$ is expressible in the form
    \begin{equation*}
        L = \Bigg [ \prod_{i=1}^k \lambda(t_i) \Bigg] exp \Bigg( - \int_0^T \lambda(u) \d u \Bigg).
    \end{equation*}
\end{theorem}

\begin{proof}
    From equation \ref{eq:JointDensHawkes} the joint density function is
    
    \begin{equation*}
        L = p(t_1, \dots, t_k) = \prod_{i = 1}^k f(t_i).
    \end{equation*}
    From equation (\ref{eq:IntFun}) we have
    
    \begin{equation*}
        \lambda(t) = \frac{f(t)}{1 - F(t)} = \frac{\frac{\mathrm{d} F(t)}{\mathrm{d} t}}{1 - F(t)} = - \frac{\mathrm{d}}{\mathrm{d} t} \log(1 - F(t))
    \end{equation*}
    therefore integrating both sides over the interval $(t_k, t)$ we have
    
    \begin{equation*}
        \int_{t_k}^t \lambda (u) \d u = - [\log(1 - F(t)) - \log(1 - F(t_k))].
    \end{equation*}
    Since the Hawkes process is a simple process, multiple arrivals cannot occur and $F(t_k) = 0$ as the probability that point $t_{k+1}$ falls on top of $t_k$ is zero, therefore we have
    
    \begin{equation}\label{logCDF}
        \int_{t_k}^t \lambda (u) \d u = - \log(1 - F(t)).
    \end{equation}
    Rearranging we have
    
    \begin{align*}
        & F(t) = 1 - \exp \Bigg( -\int_{t_k}^t \lambda(u) \d u \Bigg) \\
        & f(t) = \lambda(t) \exp \Bigg( -\int_{t_k}^t \lambda(u) \d u \Bigg)
    \end{align*}
    and the likelihood for observing a process until the time of the $k$th arrival becomes
    
    \begin{align*}
        L &= \prod_{i=1}^k f(t_i) \\
        &= \prod_{i=1}^k \lambda(t_i) \exp \Bigg( -\int_{t_{i-1}}^{t_i} \lambda(u) \d u \Bigg) \\
        &= \Bigg[ \prod_{i=1}^k \lambda(t_i) \Bigg]\exp \Bigg( -\int_{0}^{t_k} \lambda(u) \d u \Bigg).
    \end{align*}
    When a process is observed over a time period $[0, T] \subset [0, t_k]$ we will need to include in the likelihood the probability of seeing no arrivals in the interval $(t_k, T]$ so the likelihood becomes:
    
    \begin{equation*}
        L = \Bigg[ \prod_{i=1}^k f(t_i) \Bigg] (1 - F(T)) = \Bigg[ \prod_{i=1}^k \lambda(t_i) \Bigg]\exp \Bigg( -\int_{0}^{T} \lambda(u) \d u \Bigg).
    \end{equation*}
\end{proof}
Notice that this is the likelihood of an inhomogeneous Poisson process with $\Lambda(0, T] = -\int_{0}^T \lambda(u) \d u$ as in equation (\ref{eq:LikInhPoiPro}).

The likelihood for a temporal point process can also be defined in terms of the Janossy density. Here we only make a quick mention of the method. For a full explanation see Daley Chapter 7 \cite{Daley}).
Let $(A_1, \dots, A_k)$ be a partition of $[0, T]$ with $A_i$ the possible times for event $i$ and let us call $p_k$ with $k = 0, 1, \dots$ the distribution of the total number of points, with $\sum_{k=0}^\infty p_k = 1$. Let us also define for each integer $k \geq 1$ the probability distribution $\Pi_k^{sym}(\cdot)$ determining the joint distribution of the times of events in the process, given there are $k$ total events. This distribution is symmetric because we consider unordered sets and this distribution should therefore give equal weights to all $k!$. We define the Janossy measure $J_k$ as
\begin{equation*}
    J_k(A_1 \times \dots \times A_k) = k!p_k\Pi_k^{sym}(A_1 \times \dots \times A_k).
\end{equation*}
Notice that the Janossy measure is not a probability measure, but it represents the sum of the probabilities of all $k!$ permutations of $k$ points. 
When derivatives exists we can introduce the density $j_k(t_1, \dots, t_k)$ of the Janossy measure with respect to the Lebesgue measure on $(\mathbb{R})^k$ with $t_i \neq t_j$ for $i \neq j$. Then the quantity $j_n(t_1, \dots, t_n)\d t_1, \dots \d t_n$ can be interpreted as the probability that there are exactly $n$ events in the process, one in each of the infinitesimal intervals $(t_i, t_i+\d t_i)$. This interpretation connects the Janossy density to the likelihood function which can be written
\begin{equation*}
    L(t_i,\dots, t_k) = j_k(t_1, \dots, t_k | T)
\end{equation*}
were $j_k(t_1, \dots, t_k | T)$ is the local Janossy density interpreted as the probability that there are exactly $k$ events in the process before time $T$, one in each of the infinitesimal intervals.

The likelihood of a spatio-temporal self-exciting point process is obtained by treating the spatial locations as marks therefore obtain the log-likelihood (Daley and Vere-Jones (2003) Proposition 7.3.III) \cite{Daley}

\begin{proposition}
    Let $N$ be a regular marked point process on $[0, T] \times X$ where $X \subset \mathbb{R}^2$ for some positive $T$, and let $(t_1, x_1), \dots, (t_{N_g(T)}, x_{N_g(T)})$ be a realisation of $N$ over the interval $[0, T]$. Then the likelihood $L$ of such a realisation is expressible  in the form
    
    \begin{align*}
        L &= \Bigg[ \prod_{i=1}^{N_g(T)} \lambda(t_i, x_i) \Bigg]\exp \Bigg( -\int_{0}^{T} \int_X \lambda(u, k) \d u \d k \Bigg) \\
        &= \Bigg[ \prod_{i=1}^{N_g(T)} \lambda_g(t_i) \Bigg] \Bigg[ \prod_{i=1}^{N_g(T)} f(x_i | t_i) \Bigg]\exp \Bigg( -\int_{0}^{T} \lambda_g(u) \d u \Bigg)
    \end{align*}
\end{proposition}
were $\lambda_g(t)$ is the ground process and $f(x | t)$ is the conditional probability of the marks 

\begin{equation*}
    l = \sum_{i=1}^n log(\lambda(s_i, t_i)) - \int_0^T\int_X \lambda(s,t) \d s \d t
\end{equation*}

\section{Simulation of self-exciting point processes}

The simulation of the likelihood of a self-exciting spatial-temporal point process will be necessary in the application of the new SIS methodology to the RIFA model that we will encounter in future chapters.

For Hawkes processes an algorithm to generate data from the model was introduced by Ozaki in 1979 (\cite{Ozaki}). This focuses on equation (\ref{logCDF})

\begin{equation*}
    \int_{t_k}^t \lambda (u) \d u = - \log(1 - F(t)).
\end{equation*}
From this equation we can see that the next arrival time $T_{k+1}$ can be generated by the inverse transformation method, i.e. draw $U \sim \text{Unif}[0, 1]$ then $t_{k+1}$ is found solving 
\begin{equation*}
    \int_{t_k}^{t_{k+1}} \lambda(u) \d u = - \log (U).
\end{equation*}
Solving for $t_{k+1}$ can be done numerically using Newton-Raphson method, however, this entails a significant computational effort.

Ogata (\cite{Ogata81}) introduced a simulation algorithm for Point processes which is based on the thinning algorithm introduced by Lewis and Shedler (\cite{Lewis}) for inhomogeneous Poisson processes. In this process we simulate a ``faster" homogeneous Poisson Process whose rate cannot be less than $\lambda(\cdot)$ over $[0,T]$,  we then remove point probabilistically in such a way that the remaining points satisfy the intensity function $\lambda(\cdot)$.
To apply this method to an Hawkes process we notice that while the intensity for this process does not have an almost surely upper bound, it is common for the intensity to be non-increasing in period without any arrivals. In this case for $t \in [T_i, T_{i+1}]$ we have $\lambda(t) \leq \lambda (T^+_i)$ where with $T^+_i$ we mean the time just after $T_i$ when that arrival has been registered. So the value of the ``faster" process rate can be updated at each simulation.

A Simulation process based on the branching structure of a self-exciting spatial-temporal point process was introduced by Zhuang in 2004 (\cite{Zhuang}). This is a simpler and faster procedure compared to the thinning process as it does not require the repeated evaluation of $\lambda(t, s)$. This will be the algorithm we will use in our method.

As we have seen, immigrants form a homogeneous Poisson process with rate $mu$, therefore the number of immigrants over an interval $[0, T]$ is $\text{Pois}(\mu T)$. We can therefore simulate the background process using a method for homogeneous Point processes. For example, conditional on the number of immigrants $k$ the arrival times $C_1, \dots, C_k$ are distributed as the order statistics from a uniform distribution supported on $[0, T]$.

Again, we have seen that the offspring of each immigrant form an inhomogeneous Poisson process. For the $i$th immmigrant its descendants will arrive with intensity $\lambda(t-C_i)$ with $t > C_i$. The number of offspring of the immigrant $i$ will be denoted by $D_i$ and will be such that $D_i \overset{iid}{\sim}\text{Poi}(m)$ where $m$ has been defined in equation (\ref{eq:meanoffsp}). Say the offspring of the $i$th immigrant arrive at time $(C_i + E_1, C_i + E_2, \dots, C_i + E_{D_i})$. Conditioned on knowing $D_i, E_j$ are i.i.d. random variables distributed with p.d.f. $\lambda(\cdot)/m$.
