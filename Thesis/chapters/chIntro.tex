\chapter[Introduction]{Introduction}
\label{ch:Introduction}

\section{Introduction}

The problem of missing data is ubiquitous in many areas of research and while sometimes the missing data can be safely ignored during a statistical analysis, this is often not the case and it is best to impute the missing elements of the data-set making use of existing knowledge of the data and of the problem. Many methods of imputation exist and are commonly used. We will give a quick summary of the most common one in chapter \ref{ch:MissingData}.

Here we consider the specific problem of imputing missing data in the presence of incomplete but exact observations made sequentially in real time for an evolving system. We thus envisage data that consists of a series of correlated observations made sequentially in time, each of which is correct but only partially reveals the true state of the system. The main difficulty that arises in this context and that we are trying to solve here, is that data missing at one time point can be revealed at a later time point, so that imputed missing values must later be corrected in light of new information.

Our motivation for considering this problem is to facilitate analysis of invasive species, where in an ongoing surveillance program the locations of invaders are regularly being detected. For detected invaders, the location can be precisely determined, but at any given time there is an unknown number of undetected organisms, each with an unknown location. To infer the current extent of the invasion we aim to impute plausible locations of undetected individuals, but these imputations are only informed guesses, and will require constant correction as new observations come to light. In addition, new invaders are constantly being produced: the unseen state of the system is thus constantly evolving. Knowing the location of at least some of the invaders at some time $t$, we can simulate the evolution of this system. However, again the imputed locations of simulated nests will require constant correction as more information about the true state of the system becomes available.

We propose a Bayesian approach to problems of this kind that uses a new sequential importance sampling (SIS) strategy. As is typical of SIS methods, we generate a population of particles, each representing a plausible sequence of system states, and we evolve each particle at each time step according to a model of system dynamics. Again in a manner typical of SIS methods, new observations arrive in real time, and we use these observations to adjust the weights assigned to particles. However, a crucial new element in our method is that we allow missing values imputed at earlier time steps to be corrected so that they are consistent with the new observations.

Although our motivation is invasive species, problems of this kind arise in many contexts. We can envisage the approach being used to study the evolution of a species' geographic range, for both invasive and non-invasive species. The algorithm could be also applied to a variety of other missing data problems where new incomplete data is continuously acquired, like in aftershocks prediction in seismology (\cite{Seif}) or bushfire modeling (\cite{Beer}). Another potential application is in the study of the spread of infectious diseases, where observations are made in the form of diagnosed cases, but missing data in the form of undiagnosed cases is unavoidable (\cite{O'Neill}). 

In this thesis we will firstly present a quick review of Particle Filters in general and of the basic Sequential Importance Methodologies (chapter \ref{ch:PF}). We then introduce our method in chapter \ref{ch:SIS}. In order to illustrate the method and test its validity we apply the algorithm to two simple problems: a basic AR(1) model and a simple simulated river invasion. We will compare our method with analytical solutions for the AR(1) model.

The real data-set we will use is the one for the Red Imported Fire Ants (RIFA) invasion in Queensland, Australia. In Chapter \ref{ch:Hawkes} we will give an introduction of Hawkes Process and Spatio-temporal point processes that will be useful to understand chapter \ref{ch:Model} in which we will introduce a new model for the RIFA invasion based on a spatio-temporal self-exciting process.

In chapter \ref{ch:ModelMethod} will apply the new methodology to the new model for the RIFA invasion and the in chapter \ref{ch:conclusions} we will draw our conclusion and explore possible future developments of this research.

