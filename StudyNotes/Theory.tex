\documentclass[11pt,a4paper]{article}

\usepackage{amsmath}  
\usepackage{amssymb}
\usepackage{amsthm}
\usepackage{amsfonts} 
\usepackage{graphicx} 
\usepackage[usenames]{color}
\usepackage{mathtools}
\usepackage{algorithm}
\usepackage[noend]{algpseudocode}
\usepackage{float}


\DeclarePairedDelimiter{\abs}{\lvert}{\rvert}
\DeclareMathOperator{\esssupp}{ess\,supp}


 \textwidth=16cm \hoffset = -1.9cm
 \lineskip=1.5\lineskip


% MATH -----------------------------------------------------------
\newcommand{\Real}{\mathbb R}
\newcommand{\E}{\mathbb{E}}
\newcommand{\s}{\mathbb{S}}
\renewcommand{\P}{\mathbb{P}}
\newcommand{\K}{\mathbb{K}}
\newcommand{\Q}{\mathbb{Q}}
\newcommand{\eps}{\varepsilon}
\newcommand{\diag}{\mathrm{diag}}
\newcommand{\nbr}{\mathrm{nbr}}
\newcommand{\F}{\mathcal{F}}
\newcommand{\M}{\mathcal{M}}
\newcommand{\Z}{\mathcal{Z}}
\newcommand{\csimplex}{\bar{\mathcal{S}}^{d-1}}
\newcommand{\osimplex}{\mathcal{S}^{d-1}}
\newcommand{\LL}{\mathcal{L}}
\newcommand{\Hil}{\mathscr{H}}
\newcommand{\G}{\mathscr{G}}
\newcommand{\p}{\mathscr{P}}
\newcommand{\C}{\mathscr{C}}
\newcommand{\one}[1]{\mathbf{1}_{\{#1\}}}
\newcommand{\oneset}[1]{\mathbf{1}_{#1}}
\newcommand{\argmin}{\mathrm{argmin}}
\newcommand{\argmax}{\mathrm{argmax}}
\newcommand{\var}{\mathrm{Var}}
\newcommand{\cov}{\mathrm{Cov}}
\newcommand{\ind}{\mathrm{I}}
\newcommand{\D}{\mathscr{D}}
\newcommand{\Borel}{\mathscr{B}}
\newcommand{\ben}{\begin{enumerate}}
\newcommand{\een}{\end{enumerate}}
\newcommand{\ds}{\displaystyle}
\newcommand{\voila}{\hfill $\blacksquare$}
\newcommand{\Id}{\mathrm{Id}}
\renewcommand{\Re}{\mathrm{Re}}
\renewcommand{\vec}[1]{\mathbf{#1}}
\renewcommand{\d}[1]{\ensuremath{\operatorname{d}\!{#1}}}
\newcommand*\diff{\mathop{}\!\mathrm{d}}


\begin{document}

\section{Sequential Importance Sampling}



\section{Bayesian inference in a partially observed space}

Our system is in a state that can be observed without error, but where only part of the state can be observed. The state space is a topological measure space $(\Omega, \F, \lambda)$ on which a prior probability measure $\Q$ having density $q(x)$ relative to a Borel measure $\lambda$ has been defined. The partial observations are made in a second topological measure space $(\Omega', \F', \lambda')$ such that there is a measurable function $\sigma : \Omega \xrightarrow{} \Omega'$ with $\lambda' = \lambda \circ \sigma^{-1}$. Thus $\sigma$ projects $\Omega$ onto $\Omega'$ \textcolor{red}{$(\sigma(\sigma(x)) = \sigma(z) = z = \sigma(x)$)}, and only the projection is observed. A partial observation $z \in \Omega'$ fixes the value $\sigma(x) = z$, or equivalently, it consists of information that the system is in some state $x \in \sigma^{-1}(z)$.

This scenario is more general than at first appears. A more common circumstance in Bayesian inference is that a state $x \in \Omega$ determines a conditional probability measure $\P(\cdot |x)$ over the observation space $\Omega'$, rather than a specific element $\sigma(x)$. however, \textcolor{red}{for the general space where the observations that are made with error}, we may define an augmented state space $\Omega^\ast= \Omega \times \Omega'$ and then define a projection $\sigma: \Omega^\ast \xrightarrow{}\Omega'$ and then define a projection $\sigma(x, z) = z$. Thus the partially observed framework subsumes the more common circumstance as a special case.

We want to determine the posterior distribution for the state of the system $x$ in light of the observation $z$. \textcolor{red}{To do so we make use of the disintegration theorem, this needs to be proved rigorously and the citations added}

Then the posterior distribution of $x$ in light of the partial observations $z$ has density with respect to $/mu_z$ given by 
\[
    p(x|z) \propto = \begin{cases} q(x) & x \in \sigma^{-1}(z) \\ 0 &\mbox{otherwise} \end{cases}
\]
with constant of proportionality $r(z)^{-1}$ where
\[
    r(z) = \int_{\sigma^{-1}(z)} q(x)d\mu_z(x).
\]
Note that $r(z)$ is the density of the pushforward measure $\Q \circ \sigma^{-1}$ with respect to $\lambda'$

\textcolor{red}{why is this important? Is this important to note?}

\section{Bayesian inference in a partially observed space}




\textcolor{red}{do we want to prove that this method is unbiased}


\end{document}